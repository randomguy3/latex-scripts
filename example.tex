\documentclass[a4paper]{article}

\usepackage{todo}

\title{Example \LaTeX File}
\author{Alex Merry}
\date{\today}


\begin{document}

\maketitle

\section{Things Left To Do}
\label{sec:todo}

TODOs can be inline, off to the side or an item in a list.

\TODO{In the margin}
TODOs are normally placed in the margin.  This makes it easy to spot
them as you scan through the paper.  Because there is not much space in
the margin, the text should be short, such as ``flesh out'' or ``make
clearer''.  The command for this is \textbackslash{}TODO\{\textit{margin
text}\}.

\TODOinline{An inline TODO is coloured blue}.  Inline TODOs, created
with \textbackslash{}TODOinline\{\textit{text}\}, allow for a much
larger block of text to be entered.  They are often used for longer
comments, particularly for co-authors to read.

\begin{itemize}
\item When in a list environment, you may have an incomplete list
\item Inserting a \textbackslash{}TODOitem as the last
\textbackslash{}item allows you to indicate that more items should be
added, like so:
\TODOitem Insert more items
\end{itemize}


\section{Bad references}

This is a valid reference: \ref{sec:todo}.  This is invalid:
\ref{sec:does-not-exist}.  The build script will rebuild the required
number of times to make the valid reference work, but will warn about
the invalid one.

We can also insert citatations, both proper ones (\cite{valid}) and
improper ones (\cite{no-such-cite}).  The former will be found
correctly, the latter will be warned about.

\bibliographystyle{plain}
\bibliography{examplebib}

\end{document}
% vi:tw=72
